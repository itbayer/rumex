\label{kap:installation}
\chapter{Installation}

Rumex kann in zwei Stufen verwendet werden.
Einmal als lokale, private Installation%
\footnote{ 
Rumex braucht generell keine online Verbindung.
Alle benötigten öffentlichen Scripte 
werden statisch gespeichert, sind in der Installation enthalten.
Nicht einmal ein Webserver muss installiert werden.}
und als öffentliche Installation.



\label{sec:vorbereitung-des-rechners}
\section{Vorbereitung des Rechners}

Rumex ist auf ein *nix System ausgerichtet. 
Auf diesem sollten folgende Programme installiert sein:

\begin{itemize}
\item bash
\item make
\item perl
\item git
\item pandoc
\item imagemagick
\item wget
\item sitecopy
\item vim (g)vim
\item texlive
\end{itemize}

Wer mit dem Editor \texttt{vim} zurecht kommt sollte sich auch
\texttt{gvim} installieren. 
Rumex besitzt eine \texttt{gvim} Erweiterung die, 
die Arbeit bzw. die Suche nach dem richtigen Befehl am
Anfang um einiges erleichtert.

\texttt{texlive} wird nur gebraucht wenn man auch PDF Dateien erstellen möchte.

\texttt{sitecopy} wird nur gebraucht wenn man die Daten per FTP hoch laden möchte.


Bei Debian dürfte dies die nachfolgender Zeile erledigen.
\begin{verbatim}
sudo apt-get install make perl git-core \
pandoc imagemagick sitecopy gvim texlive
\end{verbatim}





\label{sec:die-lokale-installation}
\section{Die lokale Installation}

Für die Installation auf deinem Rechner musst 
zu erst das ZIP bzw. das tar.gz Archiv 
vom github Server holen und entpacken werden.

\begin{verbatim}
wget https://github.com/itbayer/rumex/archive/gh-pages.zip
unzip gh-pages.zip
\end{verbatim}

bzw.

\begin{verbatim}
wget https://github.com/itbayer/rumex/archive/gh-pages.tar.gz
tar -xzvf gh-pages.tar.gz
\end{verbatim}

Nach dem entpacken wechselt man in das Verzeichnis
\texttt{rumex-gh-pages/.rx} und starte
die Befehle \texttt{make install} und \texttt{make show}.

\begin{verbatim}
cd rumex-gh-pages/.rx/
make install
make show
\end{verbatim}

Rumex ist jetzt lokal installiert und zeigt die erste Seite an.
Auf dieser findet man eine kurz Information der Schritte die
man noch machen sollte. 
Diese Schritte sind vor allen wichtig wenn man seine Seiten
veröffentlichen möchte.



\label{sec:veroeffentlichen-der-seite}
\section{Veröffentlichen der Seite}

In der zweiten Stufen kann die Seite auf drei
Arten veröffentlicht werden.

1. Hosting auf \href{http://www.github.com}{Github}

2. Hosting auf einem Server mit git Unterstützung

3. Hosting klassisch per FTP


\label{ssec:installation-github}
\subsection{Veröffentlichung auf github}


\label{ssec:installation-gitserver}
\subsection{Veröffentlichung auf einem Server mit git Unterstützung}


\label{ssec:installation-ftpserver}
\subsection{Veröffentlichung auf einem Server mit FTP}



