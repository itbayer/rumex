\chapter{Vim Kurztasten}
Alle Rumex Kurztasten, für den Editor VIM, beginnen mit einem
\texttt{,r}\footnote{Ausnahmen bestätigen die Regel. So wurde für die
  zwei spaltige Darstellung\\\texttt{,spn2} verwendet.}. Für die
Bedienung von Rumex braucht mein nur ein paar. Viele der Kurztasten
beinhalten eine Kombination einzelner Kurztasten bzw. Befehle.

Am Anfang ist es sicher einfacher GVIM zu verwenden, da hier ein Menü
eingebaut ist welches unter anderem auch die Kurztasten anzeigt.

Wer jedoch mit der
\href{beschreibung.html\#homepage-änderung-schnell-und-immer-aktuell}{tilda}
Unterstützung arbeitet sollte sich schon ein paar Kurztasten einprägen.

\section{Textbausteine}\label{textbausteine}

Mittels dieser Kurztasten können Textbausteine eingebunden werden.

\subsection{\texttt{,rnd} (RumexNeueDatei)}\label{rnd-rumexneuedatei}

Erstellt ein neues Datei Gerüst. Dabei wird der Dateiname gleich mit
eingebunden.

\textbf{So schaut's aus}

\begin{verbatim}
% vim-kurztasten
%
%


<!--

 [vim-kurztasten](vim-kurztasten.html)
 =======================================================

Vortext INDEX

-->

Vortext INDEX und SEITE

<!-- schnipp -->

TEXT DER SEITE

<!-- vim: set ft=pandoc: -->
\end{verbatim}

\subsection{\texttt{,rnn}
(RumexNeueNachricht)}\label{rnn-rumexneuenachricht}

Erstellt einen neuen News Eintrag. Diese Kurztaste macht eigentlich nur
in der Datei \texttt{rss.rx0x} Sinn. Das Datum hinter dem Titel,
Überschrift 1, wird von Rumex gesetzt, ist also immer die aktuelle
Systemzeit beim ausführen von \texttt{,rnn}.

\begin{verbatim}
 # Neue Nachricht <!-- 2013/11/10 00:30 -->

<!--
!> Link: http://www.it-bayer.de/rumex/
!> Autor: IT-Bayer
!> Kategorie: Neues
-->

Ab hier geht die neuen Nachricht los.
\end{verbatim}

\subsection{\texttt{,rwl}
(RumexWeiterLeitung)}\label{rwl-rumexweiterleitung}

Erstellt einen neue Weiterleitungsseite.

\textbf{So schaut's aus}

\begin{verbatim}
% Weiterleitung nach ....html
%
%


<script language="javascript">
<!--
//window.location.href="....html";
// -->
</script>
\end{verbatim}

Anmerkung: Die Zeile \texttt{window.location.href="....html";} wurde
hier Kommentiert da sonst die javascript Weiterleitung greift.
Normalerweise findet man kein \texttt{//} vor der Zeile
\texttt{window.location.href=....html}.

\subsection{\texttt{,rmk}
(RumexMootitKommentar)}\label{rmk-rumexmootitkommentar}

Erstellt einen Moot.it Kommentar Abschnitt. Als Kennzeichnung wird der
Dateiname ohne \texttt{rx??} eingebaut / angehängt.

\begin{verbatim}
 # Kommentare

<a class="moot" href="https://moot.it/i/rumex/blog/vim-kurztasten"></a>
\end{verbatim}

\subsection{\texttt{,rnb}
(RumexNeuenweBlog)}\label{rnb-rumexneuenweblog}

Erstelle einen neuen Weblog Eintrag.

\begin{verbatim}
 # Rumex WebLog

_am 09.09.2013 um 15:21 schrieb IT-Bayer_

Text für den Eintrag

<div class="weblog">
Text der vorerst ausgeblendet ist.
</div>
\end{verbatim}

\section{Git Kommandos}\label{git-kommandos}

\subsection{\texttt{,rgp} (RumexGitPull)}\label{rgp-rumexgitpull}

Git pull

\subsection{\texttt{,rgs} (RumexGitStatus)}\label{rgs-rumexgitstatus}

Git status

\section{Text Formatierung}\label{text-formatierung}

\subsection{\texttt{,rff} (RumexFormatFett)}\label{rff-rumexformatfett}

Markierte Textstelle fett darstellen

\subsection{\texttt{,rfk}
(RumexFormatKursiv)}\label{rfk-rumexformatkursiv}

Markierte Textstelle kursiv darstellen

\subsection{\texttt{,rfl}
(RumexFormatListe)}\label{rfl-rumexformatliste}

Markierte Zeilen in eine Liste wandeln

\subsection{\texttt{,rfn}
(RumexFormatNummernliste)}\label{rfn-rumexformatnummernliste}

Markierte Zeile in eine Nummernliste wandeln

\subsection{\texttt{,rfb}
(RumexFormatBlock)}\label{rfb-rumexformatblock}

Markierte Zeile in einen Block wandeln

\begin{description}
\item[,rfc]
Markierte Zeile als Code darstellen
\item[,spn2]
Text mit 2er Spalten Formatierung umschließen
\item[,spn3]
Text mit 3er Spalten Formatierung umschließen
\item[,spn4]
Text mit 4er Spalten Formatierung umschließen
\end{description}

\section{Make Befehle}\label{make-befehle}

\subsection{\texttt{,rma} (RumexMakeAll)}\label{rma-rumexmakeall}

Speichert alle offenen Dokumente und erstellt daraus die HTML Datei.

\subsection{\texttt{,rmb} (RumexMakeBild)}\label{rmb-rumexmakebild}

Erstellt die unterschiedlichen Bildgrößen

\subsection{\texttt{,rmca}
(RumexMakeCleanAll)}\label{rmca-rumexmakecleanall}

Löscht alle html Dateien, alle Bildergrößen und alle xml Dateien.
beinhaltet die drei Befehle \texttt{make bclean}, \texttt{make hclean},
\texttt{make xclean}.

\subsection{\texttt{,rmcb}
(RumexMakeCleanBilder)}\label{rmcb-rumexmakecleanbilder}

Löscht alle Bildgrößen aus dem \texttt{bilder/} Verzeichnis. Es werden
nur die Bilder gelöscht die von Rumex erstellt wurden siehe
\texttt{make bilder}.

\subsection{\texttt{,rmch}
(RumexMakeCleanHtml)}\label{rmch-rumexmakecleanhtml}

Löscht alle \texttt{.html} Dateien die von Rumex erstellt wurden.

\subsection{\texttt{,rmcx}
(RumexMakeCleanXml)}\label{rmcx-rumexmakecleanxml}

Löscht alle \texttt{.xlm} Dateien die von Rumex erstellt wurden.

\subsection{\texttt{,rmcf5}
(RumexMakeCleanF8htm)}\label{rmcf5-rumexmakecleanf8htm}

Löscht alle \texttt{.htm} Dateien die mittels der Gvim Taste F5, siehe
\href{http://www.it-bayer.de/rumex/statik/index.htm}{HTML und PDF
Dateine mit pandoc und gvim erstellen}, erstellt wurden.

\subsection{\texttt{,rmh} (RumexMakeHtml)}\label{rmh-rumexmakehtml}

Erstellt die \texttt{.html} Dateien.

\subsection{\texttt{,rmi} (RumexMakeIndex)}\label{rmi-rumexmakeindex}

Erstellt die index Datei.

\subsection{\texttt{,rmm}
(RumexMakesiteMap)}\label{rmm-rumexmakesitemap}

Erstellt die sitemap Datei.

\subsection{\texttt{,rmo} (RumexMakeOnline)}\label{rmo-rumexmakeonline}

Speichert alle Dateien und stellt diese online.

\subsection{\texttt{,rmr} (RumexMakeRss)}\label{rmr-rumexmakerss}

Erstellt die rss Datei.

\subsection{\texttt{,rms} (RumexMakeSuche)}\label{rms-rumexmakesuche}

Erstellt die Rumex Suche, bzw. die Dateiliste die für die Suche
verwendet werden soll.

\section{Vorschau}\label{vorschau}

\subsection{\texttt{,rsf} (RumexShowFile)}\label{rsf-rumexshowfile}

Zeigt eine Vorschau der Seite der Datei \texttt{file:///} im
Standardbrowser.

\subsection{\texttt{,rsw} (RumexShowWww)}\label{rsw-rumexshowwww}

Zeigt eine Vorschau der Seite \textbf{Online} im Standardbrowser.

\subsection{\texttt{,rsl} (RumexShowLocal)}\label{rsl-rumexshowlocal}

Zeigt eine Vorschau der Seite auf dem eigenen Rechner. Diese Kurztaste
kann aber nur verwendet werden wenn auf dem Rechner der Apache
installiert und entsprechend eingerichtet ist.

\section{Sonstiges}\label{sonstiges}

\subsection{\texttt{,rku}}\label{rku}

Öffnet das Unterverzeichnis \texttt{.rx} in einem neuen VimTab Fenster.
Keine Ahnung warum ich diese Kurztaste so benannt habe.

\subsection{\texttt{,ros} (RumexOpenStart)}\label{ros-rumexopenstart}

Öffnet die \texttt{start.rx0s} in einen neuen VIM Tab Fenster.

\subsection{\texttt{,ror} (RumexOpenRss)}\label{ror-rumexopenrss}

Öffnet die \texttt{rss.rx0x} in einen neuen VIM Tab Fenster.
