\chapter{Vorwort}
\label{kap:vorwort}

Es gibt immer mehr Systeme zum Erstellen und Verwalten von
Internetseiten.
Probiert habe ich schon viele und mit den meisten war ich 
auch sehr zufrieden.
Was mich aber immer gestört hat waren die ständigen Updates.

\textbf{Irgendwie hat sich alles immer gebissen}

Einmal brauchte das, ich nenne es stellvertretend 
für alle Systeme, CMS eine besondere
Version eines Programms.
Dann war es wieder umgekehrt ein Update des Programms 
konnte nicht gemacht werden weil das CMS noch nicht damit 
zurecht kam.

Ab und zu kam es mir so vor als ob ich die meiste Zeit
damit verbrachte das CMS und dessen Plattform dazu zu überreden
mit einander zu arbeiten.

Ich wünschte mir immer mehr eine einfache Internetseite,
eine HTML Seite, so wie in früheren Zeiten. 

Irgendwann bin ich dann auf \texttt{markdown} und \texttt{pandoc}
gestoßen und die Rumex Idee ist entstanden.


\section{Rumex?}
\label{sec:rumex}

Rumex ist die lateinische Bezeichnung für den
\href{http://de.wikipedia.org/wiki/Ampfer}{Ampfer} und dieser taucht in
der Natur dann auf, wenn der Boden Überdüngt, Verdichtung und Beschädigt
ist. Rumex gehört zu den sogenannten Pionier Pflanzen. Er ist ein
Lückenfüller.

Genau das soll Rumex auch sein \flqq{}ein Lückenfüller\frqq{} für alle die
\ldots{} Systeme satt haben.



