\label{kap:seitenarten}
\chapter{Seiten Arten}

Die Verwendung der Seiten wird über die
Dateiendung gesteuert.
Hierzu werden vier Zeichen eingesetzt.



\begin{itemize}
\item Die ersten beiden Zeichen \verb|rx| kennzeichnen eine Rumex Datei
\item Das dritte Zeichen, immer eine Zahl, kennzeichnet ob die Datei ein
Inhaltsverzeichnis besitzt und bis welcher Ebene dieses eingebunden werden
soll. 
\item Das vierte Zeichen zeigt die Verwendung, sprich Einbindung an.
\end{itemize}





\label{sec:inhaltsverzeichnis-kennzeichnung}
\section{Inhaltsverzeichnis Kennzeichnung}

\paragraph{0} Datei ohne Inhaltsverzeichnis.

\paragraph{1} --- Datei mit Inhaltsverzeichnis bis zur ersten Ordnung.

\paragraph{2} --- Datei mit Inhaltsverzeichnis bis zur zweiten Ordnung.

\paragraph{3} --- Datei mit Inhaltsverzeichnis bis zur dritten Ordnung.

\paragraph{4} --- Datei mit Inhaltsverzeichnis bis zur vierten Ordnung.

\paragraph{5} --- Datei mit Inhaltsverzeichnis bis zur fünften Ordnung.

\paragraph{6} --- Datei mit Inhaltsverzeichnis bis zur sechsten Ordnung.





\label{sec:verwendungs-kennzeichnung}
\section{Verwendungskennzeichnung}





\section{rx0w}

\section{rx0x}

%../.rx/beschreibung.rx0w
%../.rx/datenschutz.rx0x
%../.rx/github.rx0v
%../.rx/start.rx0s
