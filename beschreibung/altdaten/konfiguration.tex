 \label{kap:konfiguration}
 \chapter{Konfiguration}


\label{sec:rumex-auf-dich-einstellen}
\section{Rumex auf dich einstellen}

Nach der Installation muss Rumex noch auf dich eingestellt werden.
Genauer gesagt sollten folgende Angabe für deine neuen 
Seite angepasst werden.

\begin{itemize}
\item Url deiner Seite
\item Impressum
\item Kopf / Fusszeile
\item Logo
\end{itemize}

Eine Kurzbeschreibung findest du, nach der Installation, auf der
ersten Seite die dir Rumex zeigt.


\label{sec:rumex-konfigurationsvariable}
\section{Rumex Konfigurationsvariable}

Die Konfiguration von Rumex wird über zwei Dateien gesteuert.
Zu einem ist das die Vorgabe Datei
\verb|.rumex/makefile/config.mk| 
und die Benutzerdatei
\verb|.rx/config.mk|, 
wobei die Einträge der Vorgabe Datei 
\verb|.rumex/makefile/config.mk|
von den Einträgen der Benutzer Datei 
\verb|.rx/config.mk| 
überschrieben werden.

Es sind auch nicht alle Einträge in der Benutzerdatei vorhanden.
Für erweiterte Einstellungen muss man Einträge aus der Vorgabedatei
in die Benutzerdatei kopieren und dementsprechend anpassen.

Nach der Installation wird eine einfache Benutzerdatei in das Verzeichnis
\verb|.rx| kopiert.




\konfig{URL}
{Definiert die URL über die die Seite erreichbar ist.
Die URL muss hier ohne abschließendes / eingegeben werden.}
{Diese Variable ist in der Vorlage vorhanden.}
{Diese Variable ist in der Benutzerdatei vorhanden.}
{\scriptsize
\begin{Verbatim}
URL="http://www.it-bayer.de/rumex"
\end{Verbatim}
}



\konfig{INDEX\_TITEL}
{Definiert den Titel der Seite.
Es können auch HTML Tags verwendet werden.}
{Diese Variable ist in der Vorlage vorhanden.}
{Diese Variable ist in der Benutzerdatei vorhanden.}
{\scriptsize
\begin{Verbatim}
INDEX_TITEL = "IT Bayer's rumex <sup style=\"font-size: .4em;\"> (github.com Version)</sup>"
\end{Verbatim}
}


\konfig{INDEX\_AUTOR}
{Definiert den Autor der Seite.}
{Diese Variable ist in der Vorlage vorhanden.}
{Diese Variable ist in der Benutzerdatei vorhanden.}
{\scriptsize
\begin{Verbatim}
INDEX_AUTOR = "Stefan Blechschmidt"!
\end{Verbatim}
}


\konfig{INDEX\_DATUM}
{Definiert den Stand, Datum der Seite.}
{Diese Variable ist in der Vorlage vorhanden.}
{}
{\scriptsize
\begin{Verbatim}
INDEX_DATUM = $(shell ls index.rx0x -l --time-style=+%Y-%m-%d | awk '{print $$6}')
\end{Verbatim}
}



\konfig{RSS\_TITEL}
{Definiert den Titel der RSS Seite.}
{Diese Variable ist in der Vorlage vorhanden.}
{Diese Variable ist in der Benutzerdatei vorhanden.}
{\scriptsize
\begin{Verbatim}
RSS_TITEL = "Neuigkeiten von rumex Baukasten"
\end{Verbatim}
}



\konfig{CSSALL}
{Definiert die CSS Datei für die Formatierung aller Seiten.}
{Diese Variable ist in der Vorlage vorhanden.}
{}
{\scriptsize
\begin{Verbatim}
CSSALL = "rxtpl/css/all.css"
\end{Verbatim}
}



\konfig{CSSSCREEN}
{Definiert die CSS Datei für die Formatierung der Bildschirmausgabe.}
{Diese Variable ist in der Vorlage vorhanden.}
{}
{\scriptsize
\begin{Verbatim}
CSSSCREEN = "rxtpl/css/screen.css"
\end{Verbatim}
}



\konfig{CSSPRINT}
{Definiert die CSS Datei für die Formatierung der Druckausgabe.}
{Diese Variable ist in der Vorlage vorhanden.}
{}
{\scriptsize
\begin{Verbatim}
CSSPRINT = "rxtpl/css/print.css"
\end{Verbatim}
}



\konfig{CSSHANDHELD}
{Definiert die CSS Datei für die Formatierung Mobiler Endgeräte.}
{Diese Variable ist in der Vorlage vorhanden.}
{}
{\scriptsize
\begin{Verbatim}
CSSHANDHELD = "rxtpl/css/handheld.css"
\end{Verbatim}
}



\konfig{SEITENBANNER}
{Definiert die CSS Datei für das Seitenbanner.}
{Diese Variable ist in der Vorlage vorhanden.}
{Diese Variable ist in der Benutzerdatei vorhanden.}
{\scriptsize
\begin{Verbatim}
SEITENBANNER = "rxtpl/img/rumex.png"
\end{Verbatim}
}



\konfig{MOOTIT}
{Kontoname für die Diskusionserweiterung mittels moot.it. 
Wird diese Variable kommentiert ist diese Erweiterung ausgeschaltet.}
{Diese Variable ist in der Vorlage vorhanden.}
{Diese Variable ist in der Benutzerdatei vorhanden.}
{\scriptsize
\begin{Verbatim}
# MOOTIT = "rumex"
\end{Verbatim}
}



\konfig{WEBLOGAUTOR}
{Diese Variable definert den Autor der Weblog Einträge}
{Diese Variable ist in der Vorlage vorhanden.}
{Diese Variable ist in der Benutzerdatei vorhanden.}
{\scriptsize
\begin{Verbatim}
WEBLOGAUTOR = "IT-Bayer"
\end{Verbatim}
}



\konfig{META\_PUBLISHER}
{Angaben zum Autor der HTML Seite.}
{Diese Variable ist in der Vorlage vorhanden.}
{Diese Variable ist in der Benutzerdatei vorhanden.}
{\scriptsize
\begin{Verbatim}
META_PUBLISHER = "IT-Bayer"
\end{Verbatim}
}



\konfig{META\_CREATOR}
{Angaben zum Autor der HTML Seite.}
{Diese Variable ist in der Vorlage vorhanden.}
{Diese Variable ist in der Benutzerdatei vorhanden.}
{\scriptsize
\begin{Verbatim}
META_CREATOR = "IT-Bayer (Stefan Blechschmidt)"
\end{Verbatim}
}



\konfig{RUMEXSUCHE}
{Verzeichnis in dem sich das JavaScript für die Suche befindet.}
{Diese Variable ist in der Vorlage vorhanden.}
{}
{\scriptsize
\begin{Verbatim}
RUMEXSUCHE = "rxtpl/js"
\end{Verbatim}
}



\konfig{FAVICON}
{Definiert das Favicon Bild der Seite}
{Diese Variable ist in der Vorlage vorhanden.}
{}
{\scriptsize
\begin{Verbatim}
FAVICON = "favicon.gif"
\end{Verbatim}
}


\konfig{SITECOPY}
{Definiert das Programm für die FTP Upload Funktion.
Ist diese Variabel kommentiert wird git verwendet.}
{Diese Variable ist in der Vorlage vorhanden.}
{Diese Variable ist in der Benutzerdatei vorhanden.}
{\scriptsize
\begin{Verbatim}
#SITECOPY = /usr/bin/sitecopy
\end{Verbatim}
}


Die TEMPLATE Dateien definieren den Seitenaufbau.
Hierzu werden vier Dateien verwendet.

\begin{enumerate}
\item HTML\_TEMPLATE
\item KOPF\_TEMPLATE
\item HEADER\_TEMPLATE
\item FUSS\_TEMPLATE
\end{enumerate}

Ist im Verzeichnis rx/ eine entsprechende Datei vorhanden wird
diese verwendet.
Ansonst wird die Datei aus dem Verzeichnis .rumex/default eingebunden.

\konfig{HTML\_TEMPLATE}
{HTML Vorlage Datei für die Seite}
{Diese Variable ist in der Vorlage vorhanden.}
{}
{\scriptsize
\begin{Verbatim}
HTML_TEMPLATE = $(shell if [ -f ../.rx/html.template ];
then echo \"../.rx/html.template\";
else echo \"../.rumex/default/html.template\";
fi)
\end{Verbatim}
}



\konfig{KOPF\_TEMPLATE}
{Definiert die Kopfdatei der Seite.}
{Diese Variable ist in der Vorlage vorhanden.}
{}
{\scriptsize
\begin{Verbatim}
KOPF_TEMPLATE = $(shell if [ -f ../.rx/kopf.html ];
then echo \"../.rx/kopf.html\";
else echo \"../.rumex/default/kopf.html\";
fi)
\end{Verbatim}
}



\konfig{HEADER\_TEMPLATE}
{Definiert die Datei in der Header Einträge der Seite vorhanden sind.}
{Diese Variable ist in der Vorlage vorhanden.}
{}
{\scriptsize
\begin{Verbatim}
HEADER_TEMPLATE = $(shell if [ -f ../.rx/header.html ];
then echo \"../.rx/header.html\";
else echo \"../.rumex/default/header.html\";
fi)
\end{Verbatim}
}



\konfig{FUSS\_TEMPLATE}
{Definiert die Datei der Informationen des Fußbereiches.}
{Diese Variable ist in der Vorlage vorhanden.}
{}
{\scriptsize
\begin{Verbatim}
FUSS_TEMPLATE = $(shell if [ -f ../.rx/fuss.html ];
then echo \"../.rx/fuss.html\";
else echo \"../.rumex/default/fuss.html\";
fi)
\end{Verbatim}
}


Jede der drei Seitentypen bekommt in HEADER der HTML
Datei die \verb|meta_robots| Variable mitgeliefert.
Anhand der nachfolgenden drei Konfigurationsvariablen
werden diese Parameter für jeden Seitentyp gesetzt.


\konfig{META\_ROBOTS\_STANDARD}
{Meta Angabe für die Standard Seiten \texttt{rx?s}. }
{Diese Variable ist in der Vorlage vorhanden.}
{}
{\scriptsize
\begin{Verbatim}
META_ROBOTS_STANDARD = "all"
\end{Verbatim}
}



\konfig{META\_ROBOTS\_VERSTECKT}
{Meta Angabe für die versteckten Seiten \texttt{rx?v}. }
{Diese Variable ist in der Vorlage vorhanden.}
{Diese Variable ist in der Benutzerdatei vorhanden.}
{\scriptsize
\begin{Verbatim}
META_ROBOTS_VERSTECKT = "noindex,nofollow,noarchive"
\end{Verbatim}
}


\konfig{META\_ROBOTS\_WEITERLEITUNG}
{Meta Angabe für die Seiten die weiterleiten \texttt{rx?w}. }
{Diese Variable ist in der Vorlage vorhanden.}
{Diese Variable ist in der Benutzerdatei vorhanden.}
{\scriptsize
\begin{Verbatim}
META_ROBOTS_WEITERLEITUNG = "noindex,follow,noarchive"
\end{Verbatim}
}



\konfig{INDEX}
{Programm zum erstellen der index.rx0x Datei}
{Diese Variable ist in der Vorlage vorhanden.}
{}
{\scriptsize
\begin{Verbatim}
INDEX="../.rumex/bin/index.pl"
\end{Verbatim}
}




\konfig{RX2RSS}
{Programm zum Erstellen der RSS Feed Datei.}
{Diese Variable ist in der Vorlage vorhanden.}
{}
{\scriptsize
\begin{Verbatim}
RX2RSS = "../.rumex/bin/rx2rss.pl"
\end{Verbatim}
}



\konfig{RSS\_FILE}
{Name der RSS Datei die in den Quelltext eingebaut wird}
{Diese Variable ist in der Vorlage vorhanden.}
{Diese Variable ist in der Benutzerdatei vorhanden.}
{\scriptsize
\begin{Verbatim}
RSS_FILE = "rss.xml"
\end{Verbatim}
}









%Kopiervorlage der Konfiguration
\begin{verbatim}





# Externe RSS URL verwenden
# 	Ist diese Variable gesetzt wird die interne
# 	RSS Verarbeitung, RSS_FILE, übersprungen.
# 	und die externe URL verwendet.

#RSS_EXTERN = "http://www.it-bayer.de/rss.xml"

\konfig{}
{}
{Diese Variable ist in der Vorlage vorhanden.}
{Diese Variable ist in der Benutzerdatei vorhanden.}
{\scriptsize
\begin{Verbatim}

\end{Verbatim}
}





#-------------------------------------------------------------------------------
# XML Sitemap
#-------------------------------------------------------------------------------
#
# Programm zum erstellen der XML Sitemap Datei
SITEMAP_XML = "../.rumex/bin/sitemap_xml.pl"

\konfig{}
{}
{Diese Variable ist in der Vorlage vorhanden.}
{Diese Variable ist in der Benutzerdatei vorhanden.}
{\scriptsize
\begin{Verbatim}

\end{Verbatim}
}


# Zieldatei der Sitemap Datei
SITEMAP_XML_FILE = "../sitemap.xml"

\konfig{}
{}
{Diese Variable ist in der Vorlage vorhanden.}
{Diese Variable ist in der Benutzerdatei vorhanden.}
{\scriptsize
\begin{Verbatim}

\end{Verbatim}
}


# Programm zum erstellen der rx0v Sitemap Datei
SITEMAP_RX = "../.rumex/bin/sitemap_rx.pl"

\konfig{}
{}
{Diese Variable ist in der Vorlage vorhanden.}
{Diese Variable ist in der Benutzerdatei vorhanden.}
{\scriptsize
\begin{Verbatim}

\end{Verbatim}
}


# Zieldatei der HTML Sitemap Datei
SITEMAP_RX_FILE = "sitemap.rx0v"

\konfig{}
{}
{Diese Variable ist in der Vorlage vorhanden.}
{Diese Variable ist in der Benutzerdatei vorhanden.}
{\scriptsize
\begin{Verbatim}

\end{Verbatim}
}



#-------------------------------------------------------------------------------
# Verzeichnis in dem sich die Bilder befinden.
#-------------------------------------------------------------------------------
BVZ="../bilder"
\konfig{}
{}
{Diese Variable ist in der Vorlage vorhanden.}
{Diese Variable ist in der Benutzerdatei vorhanden.}
{\scriptsize
\begin{Verbatim}

\end{Verbatim}
}




#-------------------------------------------------------------------------------
# Pandoc Programm
#-------------------------------------------------------------------------------
PANDOC = pandoc
# PANDOC = /usr/bin/pandoc

\konfig{}
{}
{Diese Variable ist in der Vorlage vorhanden.}
{Diese Variable ist in der Benutzerdatei vorhanden.}
{\scriptsize
\begin{Verbatim}

\end{Verbatim}
}



#-------------------------------------------------------------------------------
# Rumex Suche
#
# Programm zum Erstellen der rumexsuche_config.js
#-------------------------------------------------------------------------------
SUCHE = "../.rumex/bin/suche.pl"

\konfig{}
{}
{Diese Variable ist in der Vorlage vorhanden.}
{Diese Variable ist in der Benutzerdatei vorhanden.}
{\scriptsize
\begin{Verbatim}

\end{Verbatim}
}


#-------------------------------------------------------------------------------
# Ziel Datei für die JavaScript Suche
#-------------------------------------------------------------------------------
SUCHE_JS_CONFIG="../$(RUMEXSUCHE)/rumexsuche_config.js"
\konfig{}
{}
{Diese Variable ist in der Vorlage vorhanden.}
{Diese Variable ist in der Benutzerdatei vorhanden.}
{\scriptsize
\begin{Verbatim}

\end{Verbatim}
}




#-------------------------------------------------------------------------------
# Rumex Versionshinweis für die HTML Dateien
#-------------------------------------------------------------------------------
META_GENERATOR = "rumex "$(shell cat ../.rumex/default/version.txt)
\konfig{}
{}
{Diese Variable ist in der Vorlage vorhanden.}
{Diese Variable ist in der Benutzerdatei vorhanden.}
{\scriptsize
\begin{Verbatim}

\end{Verbatim}
}

\end{verbatim}
