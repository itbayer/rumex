\chapter{Statik Funktion}

Rumex verwendet pandocs markdown weil man damit sehr einfach und schnell
Text erstellen und in verschiedenen Formate wandeln kann. Für die
Erstellung von Denkschriften\footnote{Im Neudeutschen würde man die
  Denkschrift als Memorandum bezeichnen.} wurde zusätzlich eine, ich
nenne sie \emph{statik Funktion} eingebaut. Mit dieser Funktion ist es
möglich innerhalb eines Unterverzeichnisses verschiedenen Ausgabe
Formate zu erstellen. Zur Zeit werden, von Rumex, folgenden Formate
unterstützt:

\begin{itemize}
\itemsep1pt\parskip0pt\parsep0pt
\item
  .html
\item
  .pdf
\item
  .odt OpenDokument
\item
  .epub E-Book
\item
  .mobi E-Book (Kindle)
\end{itemize}

Erstellt werden die einzelnen Formate über
\hyperref[rumex-kurztasten]{Funktionstasten} die innerhalb
(g)vim\footnote{Die Funktionstasten sind im gvim Rumex Menü nicht
  eingebaut. Man sollte sie sich also merken :-).} zur Verfügung
stehen.\sout{Eine Besonderheit ist dass die \texttt{.htm} Datei auch
ohne die zusätzlichen Dateien wie} \sout{Bilder oder die CSS Datei
funktionieren. Alle Daten werden in die \texttt{.htm} Datei
eingebunden.}\footnote{Diese Eigenschaft wurde wieder entfernt da eine
  Datei mit sehr vielen Bildern oder gar Videos, die alle in die HTML
  Datei eingebunden werden, sehr groß wird. Dieses erhöht die Ladezeit
  der HTM Datei. Bei der Verwaltung durch git wirkt sich diese
  Eigenschaft auch ungünstig aus.}\\Auch wurde die
\hyperref[litfunk]{Literaturfunktion} von Pandoc eingebaut sodass
Verweise auf anderen Quellen in den Denkschriften verwendet werden
können.

Wer sich Rumex nicht installieren möchte aber dennoch eine einfache
Möglichkeit sucht HTML Seiten zu erstellen, kann sich einmal
\href{http://www.it-bayer.de/cirsium}{cirsium} anschauen. Cirsium ist
eine Auskopplung aus Rumex, unterteilt aber die Quelltextdatei in
verschiedene Abschnitte. Am besten einfach mal auf
\href{http://www.github.com/itbayer/cirsium}{github.com} anschauen.

\hyperdef{}{rumex-kurztasten}{\section{Die (g)vim statik Kurztaste in
Rumex}\label{rumex-kurztasten}}

Ab der Rumex Version 0.8.2 sind die Funktionstasten in Rumex enthalten.
Folgende F-Tasten wurden belegt.

\begin{description}
\item[\textbf{F5}]
Erstellt die \texttt{.htm} Datei ohne Inhaltsverzeichnis.
\item[\textbf{Shift+F5}]
Erstellt die \texttt{.htm} Datei mit Inhaltsverzeichnis bis zur Ebene 3.
\item[\textbf{Alt+F5}]
Erstellt die \texttt{.htm} Datei mit kompletten Inhaltsverzeichnis bzw.
bis Ebene 5 da LaTeX auch nicht mehr Ebenen unterstützt und die beiden
Ausgabe Format annähernd identisch sein sollten.
\item[\textbf{Ctrl+F5}]
Öffnet die \texttt{.htm} Datei.
\item[\textbf{F6}]
Erstellt die \texttt{.pdf} Datei ohne Inhaltsverzeichnis.
\item[\textbf{Shift+F6}]
Erstellt die \texttt{.pdf} Datei mit Inhaltsverzeichnis bis zur Ebene 3.
\item[\textbf{Alt+F6}]
Erstellt die \texttt{.pdf} Datei mit kompletten Inhaltsverzeichnis. Hier
ist anzumerken dass LaTeX nur Inhaltsverzeichnisse bis zur Ebene 5
unterstützt. Das HTML Format wurde entsprechend angepasst, siehe oben.
\item[\textbf{Ctrl+F6}]
Öffnet die \texttt{.pdf} Datei. Zur Zeit wird nur zathura unterstützt.
\item[\textbf{F7}]
Erstellt die restlichen Formate, \texttt{.epub}, \texttt{.odt} und
\texttt{.mobi}. Voraussetzung für das \texttt{.mobi} Format ist
\texttt{calibre}.
\item[\textbf{Ctrl-F7}]
Öffnet die Literatur Verwaltung \texttt{rumex.bib}. Voraussetzung,
\texttt{jabref} ist installiert.
\end{description}

\section{HTML Formatierung}\label{html-formatierung}

Die erzeugte HTML Datei besitzt Standardmäßig keine Formatierung bzw.
verwendet die Standard Darstellung des Browsers.

Kopf- und Fusszeile werden dadurch nicht, vom restlichen Text,
unterschieden. Auch das Inhaltsverzeichnis ist im ersten Moment als
solches nicht gleich zu erkennen. Dieses kann mit ein wenig CSS geändert
werden. Diese CSS Datei ist ab Rumex Version 0.8.2 enthalten muss aber
unter Umständen noch eingerichtet werden.

\begin{verbatim}
cd .rx
ln -s ../.rumex/default/statik.css statik.css
\end{verbatim}

\hyperdef{}{litfunk}{\section{Die Literaturverzeichnis
Funktion}\label{litfunk}}

Beim Lesen des Artikels \emph{``PDF-Dokumente schreiben mit Pandoc und
Markdown'' {[}@stenderprolinux{]}} ist mir die Idee gekommen die Rumex
\emph{statik Funktion} mit einem Literaturverzeichnis, die ja auch in
pandoc zur Verfügung steht, zu versehen.

\section{Installation pandoc manuell}\label{installation-pandoc-manuell}

Für die Verwendung der Literaturfunktion muss pandoc um das
Zusatzprogramm \texttt{pandoc-citeproc} erweitert werden. Wer Pandoc
über die Paketverwaltung installiert hat braucht hier nichts zu machen.
Wer Pandoc manuell, so wie ich, installiert hat muss dieses Programm
nachinstallieren.

Dazu erweitert man die Installationszeile um das neue Programm

\begin{verbatim}
cabal update
cabal install pandoc pandoc-citeproc
\end{verbatim}

Zu guter Letzt erstellt man noch die symbolischen Links der beiden
Programme.

\begin{verbatim}
sudo ln -s /home/USER/.cabal/bin/pandoc /usr/local/bin/.
sudo ln -s /home/USER/.cabal/bin/pandoc-citeproc /usr/local/bin/.
\end{verbatim}

\subsubsection{Nachinstallation Rumex}\label{nachinstallation-rumex}

Wer Rumex schon im Einsatz hat muss für die Erweiterung ein wenig Hand
anlegen. Zu erste holt man sich die neue Version\footnote{Die Literatur
  Erweiterung ist ab der Rumex Version 0.8.2 enthalten.} von rumex.

Dann braucht man noch drei zusätzliche Dateien im Verzeichnis
\texttt{.rx}.

\begin{itemize}
\itemsep1pt\parskip0pt\parsep0pt
\item
  \texttt{rumex.bib}
\item
  \texttt{rumex.csl}
\item
  \texttt{statik.css}
\end{itemize}

Wobei der \hyperref[literatur-stil]{Literatur Vorlage Stiel} und die CSS
Datei nur verlinkt wird. In der \texttt{rumex.bib} werden dann die
Literatur Verweise verwaltet.

\begin{verbatim}
cd .rx
touch rumex.bib
ln -s ../.rumex/default/din-1505-2.csl rumex.csl
ln -s ../.rumex/default/statik.css statik.css
\end{verbatim}

\hyperdef{}{literatur-stil}{\subsubsection{Literatur
Stil}\label{literatur-stil}}

Als Literatur Stil kommt \texttt{din-1505-2.csl} zum Einsatz. Andere
Stile findet man im git Repository
\url{https://github.com/citation-style-language/styles.git}. Als Name
für die Stil Vorlage wurde \texttt{rumex.csl} gewählt damit mit eine
Änderung des Stils einfach über den Symlink gemacht werden kann.

\subsubsection{Literatur Verwaltung}\label{literatur-verwaltung}

Für die Verwaltung der Literatur Datenbank verwende ich
\href{http://jabref.sourceforge.net/}{Jabref}.

\begin{verbatim}
sudo apt-get install jabref
\end{verbatim}

Der Aufruf des Programms wurde auch auf einen F Taste gelegt. Wer eine
anderes Programm verwenden will muss diesen entsprechend anpassen.

\subsubsection{Tipps}\label{tipps}

\begin{itemize}
\item
  Auf @wiki:bibtex findet man eine schöne Beschreibung über die
  Literatur Verwaltung mit BibTex.
\item
  Die BibTex Einträge muss man sich unter Umständen nicht einmal selber
  erstellen. Da verschiedene Seiten entsprechende Dienste anbieten.
  Gelungen finde ich die Seite von
  \url{http://www.literatur-generator.de/} aber auch auf
  \url{http://lead.to/amazon/en/?op=bt} findet man BibTeX Einträge. Zwar
  muss man diese unter Umständen noch ein wenig überarbeiten aber das
  Grund Gerüst wird einem sozusagen frei Haus geliefert.\\Wer einen
  Wikipedia Artikel zitieren dem wird unter
  \texttt{"Werkzeuge -\textgreater{} Seite zitieren"} weiter geholfen.
\item
  Für das zitieren von Internetseiten verwende ich folgende Formate,
  siehe dazu Abschnitt
  \hyperref[literaturverzeichnis]{Literaturverzeichnis}.
\end{itemize}

\begin{verbatim}
    @ELECTRONIC{ wiki:bibtex,
        author = "Wikipedia",
        title = "BibTeX --- Wikipedia{,} Die freie Enzyklopädie",
        year = "2013",
        url = "http://de.wikipedia.org/w/index.php?title=BibTeX&oldid=124228120", 
        note = "[Online; Stand 18. Dezember 2013]"
    }
\end{verbatim}

\begin{quote}
\ldots{}oder

\textbf{Achtung:} Das Formate \texttt{@WWW} wird von jabref nicht
unterstützt und gegen ein anders ausgetauscht. Bei dem Einsatz von
Jabref am besten \texttt{@ELECTRONIC} verwenden.
\end{quote}

\begin{verbatim}
    @WWW{ wiki:bibtex,
        author = "Wikipedia",
        title = "BibTeX --- Wikipedia{,} Die freie Enzyklopädie",
        year = "2013",
        url = "http://de.wikipedia.org/w/index.php?title=BibTeX&oldid=124228120", 
        note = "[Online; Stand 18. Dezember 2013]"
    }
\end{verbatim}

\section{Verwendung in- und außerhalb von
Rumex?}\label{verwendung-in--und-auuxdferhalb-von-rumex}

Innerhalb von Rumex erstellt man in einem separatem Unterverzeichnis die
entsprechende markdown Datei und dann kann es auch schon los gehen.

Außerhalb von Rumex kann man diese Funktion natürlich auch verwenden.
Mit Außerhalb meine ich Denkschriften die nicht veröffentlicht werden.
Dazu gibt es zwei Möglichkeiten.

\begin{enumerate}
\def\labelenumi{\arabic{enumi}.}
\item
  Die Datei bzw. das Verzeichnis in \texttt{.gitignore} hinterlegen.
  Somit wird diese nicht verwaltet und auch nicht, bei einem
  \texttt{make online}, hoch geladen.
\item
  Eine zweite lokale Rumex Installation die nur für Denkschriften
  verwendet wird.
\item
  \ldots{}und dann wäre da noch
  \href{https://github.com/itbayer/cirsium}{Cirsium}, eine Auskopplung
  aus Rumex, mit der man einfache html und pdf Seiten erstellen kann.
  Die Literaturverzeichnisoption ist auch enthalten. Die Formate odt,
  epub und mobi sind jedoch nicht eingebaut.
\end{enumerate}

\subsection{Setzen der Überschrift für das
Literaturverzeichnis}\label{setzen-der-uxfcberschrift-fuxfcr-das-literaturverzeichnis}

Die Überschrift für das Literaturverzeichnis muss immer am Ende des
Artikels gesetzt werden.

\textbf{Beispiel}

\begin{verbatim}
# Literaturverzeichnis
\end{verbatim}

\subsection{Einbinden von Bildern}\label{einbinden-von-bildern}

Bei dem Einbinden der Bilder muss man beachten dass die Erstellung der
statik Datei vom Verzeichnis \texttt{.rx} ausgeht.

Will man also ein Bild, dass im Ordner der Statik Datei liegt einbinden
so muss auch auf das Bild aus der Sicht des \texttt{.rx} Verzeichnisses
eingebunden werden.

Beispiel:

Das Bild liegt im Ordner \texttt{statik} somit müsste der Markdown
Befehl so aussehen.

\begin{verbatim}
![Beispielbild](../statik/beispiel.png)
\end{verbatim}

In Rumex kann man diese Funktion natürlich auch verwenden. Am besten
erstellt man sich dazu ein eigenes Unterverzeichnis und dort die Datei
\texttt{index.md} mit den Texten.

\subsection{Statik Dateien im \texttt{.rx}
Verzeichnis}\label{statik-dateien-im-.rx-verzeichnis}

Es wird sicher passieren dass man die Funktionstasten der Statik Seiten
auch bei der Bearbeitung der eigentlichen Rumex Dateien drückt. Durch
entsprechende Einträge in der \texttt{.gitignore} Datei werden solche
Dateien von einem Upload ausgeschlossen. Mit den Aufruf von
\texttt{make statikclean} können die erstellten statik Dateien
Schlussendliche aus dem \texttt{.rx} Verzeichnis entfernt werden. Dieser
Befehl wird auch bei \texttt{make clean} ausgeführt.

\subsection{Tipps}\label{tipps-1}

Das PDF Programm \texttt{zathura} hat die Eigenschaft dass wenn sich die
Datei ändert diese automatisch nachgeladen wird. Eine schöne Funktion
wenn man seinen Text, an dem man gerade arbeitet, immer wieder einmal im
Ausgabe Format betrachten will. Einfach die Taste F6 drücken, die Datei
wird auch gleich gespeichert, und mit ALT-TAB das Programm Fenster
wechseln.

Bei Format HTML geht das natürlich auch. Nur muss hier eine Erweiterung
installiert werden. Für die Browser Chromium und Firefox habe ich mit
\texttt{Auto Refresh Plus}$^{Chromium}$ bzw.
\texttt{Tab Auto Reload}$^{FireFox}$ gute Erfahrungen gemacht.

\begin{center}\rule{3in}{0.4pt}\end{center}

Die PDF Datei dieser Beschreibung kann man sich \href{index.pdf}{hier}
ansehen. Die Markdown Quelldatei kann man sich \href{index.md}{hier}
holen.

\begin{center}\rule{3in}{0.4pt}\end{center}

\hyperdef{}{literaturverzeichnis}{\section{Literaturverzeichnis}\label{literaturverzeichnis}}
